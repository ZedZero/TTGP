Gaussian processes (GPs) provide a prior over functions and allow finding complex
regularities in data. The ability of GPs to adjust the complexity of the model
to the size of the data makes them appealing to use for big datasets.
Unfortunately, standard methods for GP regression and classification scale as
$\bigO(n^3)$ with the size of the data $n$ and can not be applied when $n$
exceeds several thousands.

Numerous approximate inference methods have been proposed in the literature. Many
of these methods are based on the concept of inducing inputs (\citet{candela2005},
\citet{snelson2006}, \citet{williams2000}). These methods build a smaller set
$Z$ of $m$ points that serve to approximate the true posterior of the process
and reduce the complexity to $\bigO(nm^2 + m^3)$. \citet{titsias2009} proposed
to consider the values $u$ of the Gaussian process at the inducing inputs
as latent variables and derived a variational inference procedure to approximate
the posterior distribution of these variables. \citet{hensman2013} and
\citet{hensman2015} extended this framework by using stochastic optimization to
scale up the method and generalized it to classification problems.

Inducing input methods allow to use Gaussian processes on datasets containing
millions of examples. However, these methods are still limited in the number
of inducing inputs $m$ they can use (usually up to $10^4$). Small number of 
inducing inputs limits the flexibility of the models that can be learned with 
these methods, and does not allow to learn expressive kernel functions 
(\citet{wilson2014}). \citet{wilson2015} proposed KISS-GP
framework, which exploits the Kronecker product structure in covariance matrices
for inducing inputs placed on a multidimensional grid in the feature space.
KISS-GP has complexity $\bigO(n + D m^{1 + 1/D})$, where $D$ is the dimensionality
of the feature space. Note however, that $m$ is the number of points in a
$D$-dimensional grid and grows exponentially with~$D$, which makes the method
impractical when the number of features $D$ is larger than $4$.

In this paper, we propose TT-GP method, that can use billions of inducing
inputs and is applicable to a much wider range of datasets compared to
KISS-GP.
We achieve this by combining kernel interpolation and Kronecker algebra of 
KISS-GP with a scalable variational inference procedure. We restrict the family of
variational distributions from \citet{hensman2013} to have parameters in
compact formats. Specifically, we use Kronecker product format for the
covariance matrix $\Sigma$ and Tensor Train format (\citet{oseledets2011}) for the
expectation $\mu$ of the variational distribution over the values $u$ of the
process at inducing inputs $Z$. 

\citet{nickson2015} showed that using Kronecker
format for $\Sigma$ does not substantially affect the predictive performance
of GP regression, while allowing for computational gains. The main contribution
of our paper is combining the Kronecker format for $\Sigma$ with TT-format
for $\mu$, which, together with efficient inference procedure, allows us to
efficiently train GP models with billions of inducing inputs.

Unlike KISS-GP the proposed method has linear complexity with respect to 
dimensionality $D$ of the feature space. It means that we can apply TT-GP
to datasets that are both large and high-dimensional. Note however, that
TT-GP is constructing a grid of inducing inputs in the feature space, and
tries to infer the values of the process in all points in the grid. 
High-dimensional real-world datasets are believed to lie on small-dimensional
manifolds in the feature space, and it is impractical to try to recover the
complex non-linear transformation that a Gaussian Process defines on the 
whole feature space. Thus, we use TT-GP on raw features for datasets with
dimensionality up to $10$. For feature spaces with higher dimensionality
we propose to use kernels based on parametric projections, which can be learned
from the data.
 
\citet{wilson2016deep} and \citet{wilson2016stochastic}
demonstrated efficiency of Gaussian processes with kernels based on deep
neural networks. They used subsets of outputs of a DNN as
inputs for a Gaussian process. As the authors were using KISS-GP, they
were limited to using additive kernels, combining multiple low dimensional 
Gaussian processes. We found that DNN-based kernels are very efficient 
in combination with TT-GP. These kernels allows us to train TT-GP models
on high-dimensional datasets including computer vision tasks. Moreover,
unlike the existing deep kernel learning methods, TT-GP does not require
any changes in the GP model and allows deep kernels that produce embeddings
of dimensionality up to $10$.
