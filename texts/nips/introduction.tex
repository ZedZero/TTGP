Gaussian processes (GPs) provide a prior over functions and allow finding complex
regularities in data. The ability of GPs to adjust the complexity of the model
to the size of the data makes it apealling to use them for big datasets.
Unfortunately, standard methods for GP regression and classification scale as
$\bigO(n^3)$ with the size of the data $n$ and can not be applied when $n$
exceeds several thousands.

Numerous approximate inference methods have been proposed in the literature. Many
of these methods are based on the concept of inducing inputs (\citet{candela2005},
\citet{snelson2006}, \citet{williams2000}). These methods build a smaller set
$Z$ of $m$ points that serve to approximate the true posterior of the process
and reduce the complexity to $\bigO(nm^2 + m^3)$. \citet{titsias2009} proposed
to consider the values $u$ of the Gaussian process at the inducing inputs
as latent variables and derived a variational inference procedure to approximate
the posterior distribution of these variables. \citet{hensman2013} and
\citet{hensman2015} extended this framework by using stochastic optimization to
scale up the method and generalizing it to classification problems.

Inducing input methods allowed to use Gaussian processes on datasets containing
million of examples. However, these methods are still limited with the number
of inducing points $m$ they can use. \citet{wilson2015} proposed the KISS-GP
framework, which exploits the Kronecker product structure in covariance matrices
for inducing points placed on a multidimensional grid in the feature space.
KISS-GP has complexity $\bigO(n + D m^{1 + 1/D})$, where $D$ is the dimensionality
of the feature space. Note however, that $m$ is the number of points in a
$D$-dimensional grid and grows exponentially with $D$, which makes the method
inapplicable when the number of features $D$ is larger than $4$.

In this paper we propose an approximate inference method TT-GP, that can use billions of inducing
inputs and is applicable to higher-dimensional feature spaces than previous inference techniques. We achieve
this by combining kernel interpolation and Kronecker algebra of KISS-GP with
a scalable variational inference procedure. We restrict the family of
variational distributions from \citet{hensman2013} to have parameters in
special formats. Specifically, we use Kronecker product format for the
covariance matrix $\Sigma$ and Tensor Train format (\citet{oseledets2011}) for the
expectation $\mu$ of the variational distribution over the values $u$ of the
process at inducing inputs $Z$.

The proposed TT-GP method allows to train expressive kernel functions
on big datasets. \citet{wilson2016deep} and \citet{wilson2016stochastic}
demonstrated the efficiency of Gaussian processes with kernels based on deep
neural networks. They used subsets of the outputs of a neural network as
inputs for the Gaussian process. As the authors were using KISS-GP, they
were limited to using low dimensional Gaussian processes and had to
pretrain the network before adding the GP layer. The proposed TT-GP method
allows us to learn multidimensional embeddings and train the model end-to-end.

We demonstrate the predictive performance and scalability of the proposed TT-GP
method both with standard RBF and deep kernels on a wide range of classification
and regression tasks.

Our contributions are two-fold: we propose an approximate inference method for Gaussian Processes (Sec. \alert{?????}); we propose a way to jointly train a Gaussian Process and a neural embedding network (Ses. \alert{???}).
